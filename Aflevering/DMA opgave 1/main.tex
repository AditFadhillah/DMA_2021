\documentclass{article}
\usepackage[utf8]{inputenc}
\usepackage{caption}  
\usepackage{geometry}
\usepackage{graphicx}
\usepackage{hyperref}
\usepackage{amsmath}
\usepackage{amssymb}
\usepackage{listings}
\usepackage[english,danish]{babel}
\usepackage{tabularx}
\usepackage{fancyvrb}    
\usepackage{sectsty}

\geometry{left=3.0cm,right=3.0cm,top=2.0cm,bottom=2.0cm} 
\captionsetup{margin=10pt,font=small,labelfont=bf,labelsep=endash}
\subsectionfont{\normalsize}

\begin{document}
\begin{titlepage}
\title{\Huge DMA 2021 \\ Ugeopgave 1} 
\author{\\\\\\ \Large Hold 1 - Gruppe 7 \\\\\\ Aditya Fadhillah \\\\ Signe Dueholm Nielsen \\\\ Simon Krogh Anderson\\\\\\}
\maketitle
\thispagestyle{empty}   
\end{titlepage}


\pagenumbering{arabic} 

\section*{Del 1} \subsection*{a) Hvad returnerer exists(A, 8, 6)?}

Vi har kørt algoritmen, men her vises kun de dele af den der er relevante.

1. kørsel
\begin{Verbatim}[frame=lines,label=Algorithm exists,labelposition=topline, numbers=left]
Algorithm exists(A, n, x)
    lo = 0
    hi = 7
    while hi >= lo
        mid = 3 
        if 6 > 10 
        if 6 < 10
            hi = 2
\end{Verbatim}

2. kørsel
\begin{Verbatim}[frame=lines,label=Algorithm exists,labelposition=topline, numbers=left]
Algorithm exists(A, n, x)
    lo = 0
    hi = 2
    while hi >= lo
        mid = 1 
        if 6 > 5 
            lo = 2
\end{Verbatim}

3. kørsel
\begin{Verbatim}[frame=lines,label=Algorithm exists,labelposition=topline, numbers=left]
Algorithm exists(A, n, x)
    lo = 2
    hi = 2
    while hi >= lo
        mid = 2 
        if 6 > 6 
        if 6 < 6
        else 
            return true
\end{Verbatim}
Derfor returnerer exist(A, 8, 6) \textbf{True} \newpage



\subsection*{b) Hvad returnerer exists(A, 8, 13)?} 
  
 
1. kørsel
\begin{Verbatim}[frame=lines,label=Algorithm exists,labelposition=topline, numbers=left]
Algorithm exists(A, n, x)
    lo = 0
    hi = 7
    while hi >= lo
        mid = 3 
        if 13 > 10 
            lo = 4 
\end{Verbatim}

2. kørsel
\begin{Verbatim}[frame=lines,label=Algorithm exists,labelposition=topline, numbers=left]
Algorithm exists(A, n, x)
    lo = 4
    hi = 7
    while hi >= lo
        mid = 5 
        if 13 > 16
        else if 13 < 16
            hi = 4 
\end{Verbatim}

3. kørsel
\begin{Verbatim}[frame=lines,label=Algorithm exists,labelposition=topline, numbers=left]
Algorithm exists(A, n, x)
    lo = 4
    hi = 4
    while hi >= lo
        mid = 4 
        if 13 > 12
            lo = 5 
\end{Verbatim}

3. kørsel
\begin{Verbatim}[frame=lines,label=Algorithm exists,labelposition=topline, numbers=left]
Algorithm exists(A, n, x)
    lo = 5
    hi = 4
    while hi >= lo
    return false 
\end{Verbatim}

Derfor returnerer exists(A, 8, 13) \textbf{False} \newpage

\subsection*{c) Hvad returnerer exists(A, 5, 16)?} 

 
1. kørsel
\begin{Verbatim}[frame=lines,label=Algorithm exists,labelposition=topline, numbers=left]
Algorithm exists(A, n, x)
    lo = 0
    hi = 4
    while hi >= lo
        mid = 2
        if 16 > 6 
            lo = 3
\end{Verbatim}

2. kørsel
\begin{Verbatim}[frame=lines,label=Algorithm exists,labelposition=topline, numbers=left]
Algorithm exists(A, n, x)
    lo = 3
    hi = 4
    while hi >= lo
        mid = 3
        if 16 > 10 
            lo = 4
\end{Verbatim}

3. kørsel
\begin{Verbatim}[frame=lines,label=Algorithm exists,labelposition=topline, numbers=left]
Algorithm exists(A, n, x)
    lo = 4
    hi = 4
    while hi >= lo
        mid = 4
        if 16 > 12 
            lo = 5
\end{Verbatim}

4. kørsel
\begin{Verbatim}[frame=lines,label=Algorithm exists,labelposition=topline, numbers=left]
Algorithm exists(A, n, x)
    lo = 5
    hi = 4
    while hi >= lo
    return false
\end{Verbatim}

Derfor returnerer exists(A, 5, 16) \textbf{False} \newpage


\subsection*{d) Lav en tabel over hvilke værdier de variable lo, mid og hi antager i hver iteration af while-loopet, lige efter at mid er blevet udregnet, når man kalder exists(A, 8, 14)} 

\begin{center}
\begin{tabular}{ | m{5em} | m{1cm}| m{1cm}| m{1cm}|m{1cm}| } 
  \hline
  Iteration & 1 & 2 & 3 & 4 \\ 
  \hline
  lo & 0 & 4 & 4 & 5 \\ 
  \hline
  mid & 3 & 5 & 4 & ? \\ 
  \hline
  hi & 7 & 7 & 4 & 4 \\ 
  \hline
\end{tabular}
\end{center}


\section*{Del 2} \subsection*{a) Forklar med jeres egne ord, hvad funktionen exists gør.} 

Funktionen Exist undersøger om et givent tal eksisterer i arrayet. Hvis det eksisterer, returnerer den true og hvis det ikke eksisterer returnerer den false. 

\section*{Del 3} \subsection*{a) Hvis x er et tal, der ikke findes i A, kan exists(A, n, x) så returnere True? Begrund dit svar. Hvis ja, giv et eksempel på, hvordan dette kan ske.} 

Nej, den kan ikke returnere true ved en fejl, da der stadig vil være to tal rundt om der enten er større eller mindre da x jo ikke indgår i arrayet.
\
\subsection*{b) Hvis x er et tal, der findes i A, kan exists(A, n, x) så returnere False? Begrund dit svar. Hvis ja, giv et eksempel på, hvordan dette kan ske.} 

Ja, tag f.eks arrayet A = [1,5,6,10,12,16,17,43]. Hvis vi flytter tallet 6 så A = [1,5,10,12,16,17,6,43] nu kan vi køre algoritmen der starter med en mid-værdi på 3, og da 6 er mindre end 12 går vi mod venstre, og vil derfor aldrig ramme tallet 6 der står på index 6. Derfor returnerer den false.


\section*{Del 4} \subsection*{a) Når n= 80, hvor mange gange kan while-loopet så højst køres igennem ved et kaldtilexists(A, n, x)?} 
Hver gang vi køre et loop, så halveres størrelsen af arrayet. Det vil sige at den maksimale køretid er angivet ved $n/2^{Antal iteration}$. For at finde ud af hvor mange gange while-loopet kan køre regner vi log2(80) = 6.32 men da man ikke kan køre kommatal i loops skal vi altid runde ned ved halve, vi kan skrive hele regnestykket ud sådan her:\\\
\newline
1. 80/2 = 40 \newline
2. 40/2 = 20 \newline
3. 20/2 = 10 \newline
4. 10/2 = 5 \newline
5. 5/2 = 2.5 \newline
runder ned til 2
6. 2/2 = 1 \newline

Det vil sige vi kører loopet 6 gange. 

While loopet kan altså køre maksimalt 6 gange når n = 80 


\end{document}


